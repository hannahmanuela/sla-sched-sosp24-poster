%-------------------------------------------------------------------------------
\section{Current State}
%-------------------------------------------------------------------------------

Implementing this design required changing linux slightly. Linux's EEVDF
scheduler has a strong tenant of avoiding unfairness, and as a result implements
eligibility solely as a function of how much time a process had gotten compared
to other processes. This means that (assuming equally weighted processes), after
a tick of running, a process becomes ineligible, because it got more time that
it should have (in an ideal system the tick would have been shared equally among
all processes).

In \sysname, however, being able to be unfair for long periods of time is key:
jobs with short deadlines need to get absolute priority over jobs that have
later deadlines. To achieve this goal we had to modify Linux's EEVDF
implementation, to include the notion of request-based eligibility.

We developed a prototype application that follows the scheme of the website:
four different types of jobs, with deadlines ranging from 15ms to 6s, and
maximum computes from 12ms to 4s. Figure~\ref{fig:graph} shows the results on a
single machine, with and without the change to linux. 

Overall the graph shows success: being temporarily unfair to longer processes
that had a large amount of slack allowed the shorter processes with less slack
to consistently complete on time.