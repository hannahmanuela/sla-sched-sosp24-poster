%-------------------------------------------------------------------------------
\section{Open Questions}
\label{s:open}
%-------------------------------------------------------------------------------

There remain open questions from the challenges section that represent areas 
for future work.

\textbf{Redaction and upgrade specifications}
% 
Our prototype
currently has a limited set of redaction primitives, and our design only supports
redactions that select and apply to one row at a time. This prohibits some
forms of redactions, for example redactions for a joined row, redactions over aggregations.
Or perhaps a developer testing new application features should interact with a
version of the database that contains fully synthetic data that follows
realistic distributions of values. Or a
data analyst should be allowed to perform aggregate queries over data they
cannot query individually. These use cases require more
elaborate redactions (and perhaps in places during query execution other than
just on top of the input tables) which may be more expensive to apply, so
supporting them will require us to come up with a theory of how redactions
compose, and an execution strategy that applies them efficiently and correctly.

\textbf{Avoiding overprompting}
% 
In addition to allowing some upgrades to be authorized by data subjects, 
another way to lessen over-prompting is that the database could introduce a notion of rate-limited,
audited upgrade templates that authorizing employees can sign once, and which automatically
authorize a particular set of upgrades.
The template could include not only upgrade content, but also the user role and
other attributes that are conditions for automatic approval. However, templates also reduce 
the oversight built into the system design, and a compromised employee might abuse them to exfiltrate data.
Making such templates and their approval user-friendly is an interesting direction for further research,
and will require new APIs and user interfaces.

\textbf{Writes}
% 
Our current design chooses to only allow writes if the data to be written
would be unredacted for the writer. This has the downside that it might 
lead to frequently failing writes and consequent prompts to
the client to upgrade, burdening both the client and authorizer. With this
approach to writes, a Funhouse-like database must find a usable solution for
write upgrades that avoids overprompting.

