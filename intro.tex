%-------------------------------------------------------------------------------
\section{Introduction}
%-------------------------------------------------------------------------------

There is a mismatch between what developers care about (latencies) and what they
are required to give providers (resource reservations). Developers use Service
Level Objectives (SLOs) between teams and Service Level Agreements (SLAs)
towards customers, each of which represent promises about maximal latencies of
the API that team/company is in charge of (add citations). 

This mismatch is problematic on two levels: one is that the binary of
latency-critical and best effort is a false dichotomy, and the other is that
translating latencies into reservations is difficult. 

Imagine a web developer, whose website has four different types of work it has
to do: \\
(1) load static pages (eg the homepage) - shortest and very time ciritcal \\
(2) load dynamic pages (eg a users profile page) - slightly longer and less time
critcial \\
(3) foreground data processing (eg processing a user uploaded file of image) -
require a fair amount of processing but is still user-facing and thus latency sensitive \\
(4) background data processing (eg updating a data warehouse) - run overnight
and just needs to finish by morning.

The only candidate for true best effort work is (4), and even that will pose a
problem if it's not done by the time business picks up the next day. For the
other three, the translation to a reservation system then happens offline, and
requires the developer to make estimations about peak load as well as how much
they are willing to let latencies spike (for example, (3) might be fine to run a
bit slower when load is high but (1) should remain completely impervious to load
spikes). This estimation process also incentivizes the developer to overestimate
- low utilization is much less of a problem to them than missing deadlines. 

This in turn poses a problem for providers, for whom high utilization means more
work they are able to run and thus more money. Bin-packing latency critical
work, as well as scheduling in best effort work opportunistically to use
resources guaranteed to latency critical jobs in time that they might not be
using it, is a hard problem that many systems work on.

This submission presents \textit{\sysname}, a SLA-based scheduling system that
avoids the problem alltogether. Rather than deal with the lossy metric of
reservations and the false binary of latency critcial and best effort,
\sysname{} creates a spectrum of criticality. It does this by making deadline
and maximum compute time the central scheduling mechanism.\hmng{don't love this
sentence yet, I'd like to bring it to a point but this isn't quite it}
