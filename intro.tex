%-------------------------------------------------------------------------------
\section{Introduction}
%-------------------------------------------------------------------------------

Developers use Service Level Objectives (SLOs) between teams and Service Level
Agreements (SLAs) towards customers, each of which represent guarantees about
uptime and maximal latencies of the API that team/company is in charge
of~\cite{awssla}. Monitoring systems watch performance and page on-call
developers if these guarantees are not being met~\cite{cloudwatch}.

Cloud-compute providers, on the other hand, have as their central metric of
value utilization. The more work they can pack onto their machines, the more
money they make. The interface cloud providers offer in order to acheive this is
well-known: latency-critical (LC) applications have concrete reservations, and
jobs with no immediate deadline can be run as cheaper best-effort (BE) tasks.
Cloud providers then bin-pack LC work, and schedule in BE work
opportunistically. This is not a simple proposition, but researchers have built
systems that do well at keeping utilization high, given enough best-effort
work~\cite{caladan}.


For a developer to translate their requirements into the cloud providers'
interface requires two steps: pick a category (LC or BE), and if the its the
former, generate a concrete reservation. Both of these steps are difficult: some
work might not be completely LC or BE, and reservations require the developer to
make estimations about peak load, and in fact incentivizes them to
overestimate~\cite*{overprovision} --- directly in opposition of the cloud
providers goals.

Imagine a web developer, whose website has four different types of work it has
to do: \\
(1) load static pages (eg the homepage) --- shortest and very time critical \\
(2) load dynamic pages (eg a users profile page) --- slightly longer and less
time critcial \\
(3) foreground data processing (eg processing a user uploaded file of image) ---
requires a fair amount of processing but still user-facing and thus latency
sensitive \\
(4) background data processing (eg updating a data warehouse) --- runs overnight
and just needs to finish by morning.

The only candidate for BE is (4). The other three are user-facing and as such
are LC and require reservations. How much these reservations are
over-provisioned will reflect the developers interest in keeping low latency: it
is critical that the homepage load time be low and constant, whereas processing
a user upload might be fine to take more time during high load.

This submission presents \textit{\sysname}, a SLA-based scheduling system.
Central to \sysname{} is the observation that the original priorities of
utilization and latency don't have to be opposing. In order to help align
incentives, \sysname{} changes the interface with which resource requirements
are communicated: \sysname{} makes deadlines and maximum compute times the
central scheduling mechanism. \sysname{} thereby side-steps the LC/BE binary,
and makes the developer's requirements explicit to the scheduler. The cloud
provider can use the information of how long a process will run to make better
placement decisions.
