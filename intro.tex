%-------------------------------------------------------------------------------
\section{Introduction}
%-------------------------------------------------------------------------------


There is a fundamental mismatch between what developers care about (latencies)
and what they are required to give providers (resource reservations). Developers
use Service Level Objectives (SLOs) between teams and Service Level Agreements
(SLAs) towards customers, each which represent promises about maximal latencies
of the API that team/company is in charge of (add citations). The translation to
a reservation system then happens offline, and requires the developer to make
estimations about peak load.  It also incentivizes them to overestimate - low
utilization is much less of a problem than missing deadlines. 

This in turn poses a problem for providers, for whom high utilization means more
work they are able to run and thus more money. Bin-packing latency critical
work, as well as scheduling in best effort work opportunistically to use
resources guaranteed to latency critical jobs in time that they might not be
using it, is a hard problem that many systems work on.

What this work tries to do is avoid the problem alltogether. Rather than deal
with the lossy metric of reservations and the false binary of latency critcial
and best effort, this project attempts to create a spectrum of work, each job
being characterized by a deadline and a maximum compute time.

Work is submitted as a handler function, basically a binary. Attached is
metadata that includes the maximum execution time as well as a deadline. When a
request for this handler comes in, it is initially routed to the relevant shard
of a global scheduler. Depending on the amount of slack (difference between
deadline and max compute) the global scheduler will either immediately place the
job, or will probe a dynamic number of machines in order to find a good match.
Each machine has a dispatcher, which is in communication with a shard of the
global scheduler. The dispatcher accepts and runs work, as well as doing
back-of-the-envelope computation when probed about whether a new job would fit. 

The admission control is pessismistc, and assumes that every process will use
the maximum compute time. The dispatcher looks at the amount of slack each
process has (given the original slack and how much time it has already spent
waiting), and then judges whether a new process would still allow it to meet all
the max compute guarantees. Once admitted to a machine, work is run using an
approximation of Earliest Deadline First scheduling. This is acheived via a
slight abuse of linux' new scheduling algorithm, Earliest Eligible Virtual
Deadline First (EEVDF). It also required editing linux in a small way to make
its behavior closer to the original paper (from which the linux implementation
diverges in meaningful ways).

