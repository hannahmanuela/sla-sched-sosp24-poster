%-------------------------------------------------------------------------------
\section{Introduction}
%-------------------------------------------------------------------------------

Developers use Service Level Objectives (SLOs) between teams and Service Level
Agreements (SLAs) towards customers, each of which represent guarantees about
uptime and maximal latencies of the API that team/company is in charge
of~\cite{awssla}. Monitoring systems watch performance and page on-call
developers if these guarantees are not being met~\cite{cloudwatch}.

On the other hand, the interface to many scheduling frameworks, such as
Kubernetes and many research systems~\cite{caladan}, offers developers two
options: latency-critical (LC) applications have concrete reservations, and jobs
with no immediate deadline can be run as cheaper best-effort (BE) tasks. The
scheduler bin-packs LC work, and schedules in BE work opportunistically.


For a developer to translate their requirements into this interface requires two
steps: pick a category (LC or BE), and if the its the former, generate a
concrete reservation. Both of these steps are difficult: some work might not be
completely LC or BE, and reservations require the developer to make estimations
about peak load, and in fact incentivizes them to
overestimate~\cite*{overprovision} --- making maintaining high utilization even
harder.

Imagine a web developer, whose website has four different types of work it has
to do: \\
(1) load static pages (eg the homepage) --- shortest and very time critical; 
(2) load dynamic pages (eg a users profile page) --- slightly longer and less
time critcial;
(3) foreground data processing (eg processing a user uploaded file of image) ---
requires a fair amount of processing but still user-facing and thus latency
sensitive;
(4) background data processing (eg updating a data warehouse) --- runs overnight
and just needs to finish by morning.

The only candidate for BE is (4). The other three are user-facing and as such
are LC and require reservations. How much these reservations are
over-provisioned will reflect the developers interest in keeping low latency: it
is critical that the homepage load time be low and constant, whereas processing
a user upload might be fine to take more time during high load.

This submission presents \textit{\sysname}, a SLA-based scheduling system.
Central to \sysname{} is the observation that the priorities of utilization and
latency don't have to be opposing.\ \sysname{} changes the interface with which
resource requirements are communicated: it makes \textit{deadlines} and
\textit{maximum compute times} the central metrics provided, thereby
side-stepping the LC/BE binary, and making the developer's requirements explicit
to the scheduler. The scheduler can use the information of how long a process
will run to make better placement decisions.

In \sysname{}, developers submit jobs, attached with the maximum execution time
as well as a deadline. In the website example, rather than estimate load for
each job, the resources required to run it, then add 20\% padding, developers
can simply give the scheduler the handler for each endpoint, and attach to each
its deadline and an experiential maximum compute time.