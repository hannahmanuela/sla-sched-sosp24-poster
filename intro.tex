%-------------------------------------------------------------------------------
\section{Introduction}
%-------------------------------------------------------------------------------

Developers use Service Level Objectives (SLOs) between teams and Service Level
Agreements (SLAs) towards customers, which represent guarantees about uptime and
maximal latencies of the API that team/company is in charge of~\cite{awssla}.
Monitoring systems watch performance and page on-call developers if these
guarantees are not being met~\cite{cloudwatch}.

However, the interface to many scheduling frameworks, such as
Kubernetes~\cite{kubernetes} and many research systems~\cite{caladan}, offers
developers two options: latency-critical (LC) applications have concrete
reservations, and jobs with no immediate deadline can be run as best-effort (BE)
tasks. The scheduler bin-packs LC work, and runs BE work opportunistically.


For a developer to translate their requirements into this interface requires two
steps: pick a category (LC or BE), and, if the its the former, generate a
concrete reservation. Both of these steps are difficult: some work might not be
completely LC or BE, and reservations require the developer to make estimations
about peak load, and in fact incentivizes them to
overestimate~\cite*{overprovision} --- making achieving high utilization hard.

Imagine a web developer, whose website has four different types of work it has
to do: \\
(1) load static pages (eg the homepage) --- shortest and very time critical; 
(2) load dynamic pages (eg a users profile page) --- slightly longer and less
time critical;
(3) foreground data processing (eg processing a user uploaded file of image) ---
requires a fair amount of processing but still user-facing and thus latency
sensitive;
(4) background data processing (eg updating a data warehouse) --- runs overnight
and just needs to finish by morning.

The only candidate for BE is (4). The other three are user-facing and as such
are LC and require reservations. But among 1-3 there are levels of
criticality: it is essential that the homepage load time be low and constant,
but processing a user upload can take more time during high load.

This submission presents \textit{\sysname}, a SAL-based scheduling system.
Central to \sysname{} is the observation that the priorities of utilization and
latency don't have to be opposing.\ \sysname{} changes the interface with which
resource requirements are communicated: it makes \textit{deadlines} and
\textit{maximum compute times} the central metrics provided, thereby
side-stepping the LC/BE binary, and making the developer's requirements explicit
to the scheduler.

In \sysname{}, developers submit jobs, attached with the maximum execution time
as well as a deadline. In the website example, rather than estimate load for
each job, the resources required to run it, then add 20\% padding, developers
can simply give the scheduler the handler for each endpoint, and attach to each
its deadline and an experiential maximum compute time.