%-------------------------------------------------------------------------------
\section{Current State}
%-------------------------------------------------------------------------------

Implementing this design required changing linux slightly. Linux's EEVDF
scheduler has a strong tenant of avoiding unfairness, and as a result implements
eligibility solely as a function of how much time a process had gotten compared
to other processes. This means that (assuming equally weighted processes), after
a tick of running, a process becomes ineligible, because it got more time that
it should have (in an ideal system the tick would have been shared equally among
all processes).

In \sysname, however, being able to be unfair for long periods of time is key:
jobs with short deadlines need to get absolute priority over jobs that have
later deadlines.

We have developed a prototype application that follows the scheme of the
website: four different types of jobs, with deadlines ranging from 15ms to 6s,
and maximum computes from 12ms to 4s. Figure~\ref{fig:graph} shows the results
on a single machine, with and without the change to linux. 

The three points above the 100$\%$ latency line in subfigure~\ref{fig:graph:new}
are anomalies that should not exist --- in a pessimistic admission control setting
all jobs should finish on time. We traced the issue back to how linux deals with
dequeueing processes with large amounts of lag, and are currently working on
acheiving the desired behavior (processes with the smallest deadline always run
first).